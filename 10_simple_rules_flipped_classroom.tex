% Template for PLoS
% Version 3.5 March 2018
%
% % % % % % % % % % % % % % % % % % % % % %
%
% -- IMPORTANT NOTE
%
% This template contains comments intended 
% to minimize problems and delays during our production 
% process. Please follow the template instructions
% whenever possible.
%
% % % % % % % % % % % % % % % % % % % % % % % 
%
% Once your paper is accepted for publication, 
% PLEASE REMOVE ALL TRACKED CHANGES in this file 
% and leave only the final text of your manuscript. 
% PLOS recommends the use of latexdiff to track changes during review, as this will help to maintain a clean tex file.
% Visit https://www.ctan.org/pkg/latexdiff?lang=en for info or contact us at latex@plos.org.
%
%
% There are no restrictions on package use within the LaTeX files except that 
% no packages listed in the template may be deleted.
%
% Please do not include colors or graphics in the text.
%
% The manuscript LaTeX source should be contained within a single file (do not use \input, \externaldocument, or similar commands).
%
% % % % % % % % % % % % % % % % % % % % % % %
%
% -- FIGURES AND TABLES
%
% Please include tables/figure captions directly after the paragraph where they are first cited in the text.
%
% DO NOT INCLUDE GRAPHICS IN YOUR MANUSCRIPT
% - Figures should be uploaded separately from your manuscript file. 
% - Figures generated using LaTeX should be extracted and removed from the PDF before submission. 
% - Figures containing multiple panels/subfigures must be combined into one image file before submission.
% For figure citations, please use "Fig" instead of "Figure".
% See http://journals.plos.org/plosone/s/figures for PLOS figure guidelines.
%
% Tables should be cell-based and may not contain:
% - spacing/line breaks within cells to alter layout or alignment
% - do not nest tabular environments (no tabular environments within tabular environments)
% - no graphics or colored text (cell background color/shading OK)
% See http://journals.plos.org/plosone/s/tables for table guidelines.
%
% For tables that exceed the width of the text column, use the adjustwidth environment as illustrated in the example table in text below.
%
% % % % % % % % % % % % % % % % % % % % % % % %
%
% -- EQUATIONS, MATH SYMBOLS, SUBSCRIPTS, AND SUPERSCRIPTS
%
% IMPORTANT
% Below are a few tips to help format your equations and other special characters according to our specifications. For more tips to help reduce the possibility of formatting errors during conversion, please see our LaTeX guidelines at http://journals.plos.org/plosone/s/latex
%
% For inline equations, please be sure to include all portions of an equation in the math environment.  For example, x$^2$ is incorrect; this should be formatted as $x^2$ (or $\mathrm{x}^2$ if the romanized font is desired).
%
% Do not include text that is not math in the math environment. For example, CO2 should be written as CO\textsubscript{2} instead of CO$_2$.
%
% Please add line breaks to long display equations when possible in order to fit size of the column. 
%
% For inline equations, please do not include punctuation (commas, etc) within the math environment unless this is part of the equation.
%
% When adding superscript or subscripts outside of brackets/braces, please group using {}.  For example, change "[U(D,E,\gamma)]^2" to "{[U(D,E,\gamma)]}^2". 
%
% Do not use \cal for caligraphic font.  Instead, use \mathcal{}
%
% % % % % % % % % % % % % % % % % % % % % % % % 
%
% Please contact latex@plos.org with any questions.
%
% % % % % % % % % % % % % % % % % % % % % % % %

\documentclass[10pt,letterpaper]{article}
\usepackage[top=0.85in,left=2.75in,footskip=0.75in]{geometry}

% amsmath and amssymb packages, useful for mathematical formulas and symbols
\usepackage{amsmath,amssymb}

% Use adjustwidth environment to exceed column width (see example table in text)
\usepackage{changepage}

% Use Unicode characters when possible
\usepackage[utf8x]{inputenc}

% textcomp package and marvosym package for additional characters
\usepackage{textcomp,marvosym}

% cite package, to clean up citations in the main text. Do not remove.
\usepackage{cite}

% Use nameref to cite supporting information files (see Supporting Information section for more info)
\usepackage{nameref,hyperref}

% line numbers
\usepackage[right]{lineno}

% ligatures disabled
\usepackage{microtype}
\DisableLigatures[f]{encoding = *, family = * }

% color can be used to apply background shading to table cells only
\usepackage[table]{xcolor}

% array package and thick rules for tables
\usepackage{array}

% create "+" rule type for thick vertical lines
\newcolumntype{+}{!{\vrule width 2pt}}

% create \thickcline for thick horizontal lines of variable length
\newlength\savedwidth
\newcommand\thickcline[1]{%
  \noalign{\global\savedwidth\arrayrulewidth\global\arrayrulewidth 2pt}%
  \cline{#1}%
  \noalign{\vskip\arrayrulewidth}%
  \noalign{\global\arrayrulewidth\savedwidth}%
}

% \thickhline command for thick horizontal lines that span the table
\newcommand\thickhline{\noalign{\global\savedwidth\arrayrulewidth\global\arrayrulewidth 2pt}%
\hline
\noalign{\global\arrayrulewidth\savedwidth}}


% Remove comment for double spacing
%\usepackage{setspace} 
%\doublespacing

% Text layout
\raggedright
\setlength{\parindent}{0.5cm}
\textwidth 5.25in 
\textheight 8.75in

% Bold the 'Figure #' in the caption and separate it from the title/caption with a period
% Captions will be left justified
\usepackage[aboveskip=1pt,labelfont=bf,labelsep=period,justification=raggedright,singlelinecheck=off]{caption}
\renewcommand{\figurename}{Fig}

% Use the PLoS provided BiBTeX style
\bibliographystyle{plos2015}

% Remove brackets from numbering in List of References
\makeatletter
\renewcommand{\@biblabel}[1]{\quad#1.}
\makeatother



% Header and Footer with logo
\usepackage{lastpage,fancyhdr,graphicx}
\usepackage{epstopdf}
%\pagestyle{myheadings}
\pagestyle{fancy}
\fancyhf{}
%\setlength{\headheight}{27.023pt}
%\lhead{\includegraphics[width=2.0in]{PLOS-submission.eps}}
\rfoot{\thepage/\pageref{LastPage}}
\renewcommand{\headrulewidth}{0pt}
\renewcommand{\footrule}{\hrule height 2pt \vspace{2mm}}
\fancyheadoffset[L]{2.25in}
\fancyfootoffset[L]{2.25in}
\lfoot{\today}

%% Include all macros below

\newcommand{\lorem}{{\bf LOREM}}
\newcommand{\ipsum}{{\bf IPSUM}}

%% END MACROS SECTION


\begin{document}
\vspace*{0.2in}

% Title must be 250 characters or less.
\begin{flushleft}
{\Large
\textbf\newline{Ten simple rules for implementing a flipped classroom} % Please use "sentence case" for title and headings (capitalize only the first word in a title (or heading), the first word in a subtitle (or subheading), and any proper nouns).
}
\newline
% Insert author names, affiliations and corresponding author email (do not include titles, positions, or degrees).
\\
Pierre Poulain\textsuperscript{1*},
Mickael Bertrand\textsuperscript{2},
Héloise Dufour\textsuperscript{3,4},
Antoine Taly\textsuperscript{5,6*},
%Name5 Surname\textsuperscript{2\ddag},
%Name6 Surname\textsuperscript{2\ddag},
%Name7 Surname\textsuperscript{1,2,3*},

%with the Lorem Ipsum Consortium\textsuperscript{\textpilcrow}
\bigskip
\textbf{1} Université de Paris, CNRS, Institut Jacques Monod, F-75013, Paris, France
\\
\textbf{2} Éducation Nationale, Académie de Dijon, Lycée Anna Judic, F-21140, Semur-en-Auxois, France
\\
\textbf{3} Cercle FSER, F-75007, Paris, France
\\
\textbf{4} Inversons la Classe !, F-75019, Paris, France
\\
\textbf{5} CNRS, Université de Paris, UPR 9080, Laboratoire de Biochimie Théorique, F-75005, Paris, France
\\
\textbf{6} Institut de Biologie Physico-Chimique, Fondation Edmond de Rothschild, PSL Research University, F-75005, Paris, France
\\
\bigskip

% Insert additional author notes using the symbols described below. Insert symbol callouts after author names as necessary.
% 
% Remove or comment out the author notes below if they aren't used.
%
% Primary Equal Contribution Note
%\Yinyang These authors contributed equally to this work.

% Additional Equal Contribution Note
% Also use this double-dagger symbol for special authorship notes, such as senior authorship.
%\ddag These authors also contributed equally to this work.

% Current address notes
%\textcurrency Current Address: Dept/Program/Center, Institution Name, City, State, Country % change symbol to "\textcurrency a" if more than one current address note
% \textcurrency b Insert second current address 
% \textcurrency c Insert third current address

% Deceased author note
%\dag Deceased

% Group/Consortium Author Note
%\textpilcrow Membership list can be found in the Acknowledgments section.

% Use the asterisk to denote corresponding authorship and provide email address in note below.
* taly@ibpc.fr, pierre.poulain@u-paris.fr

\end{flushleft}
% Please keep the abstract below 300 words
\section*{Abstract}

This article aims to share some tips for implementing a flipped classroom, with examples from biology and computational biology. 
The proposed rules are:

\begin{itemize}
\item Invert time, and say it.
\item Identify (or create) resources for learning in autonomy.
\item Be explicit in the acquisition of knowledge, and foster students' autonomy.
\item Explore active learning in class.
\item Develop skills of cooperation and sharing.
\item Evaluate.
\item Differentiate.
\item Take care of the logistics and your posture.
\item Document your flipped classroom.
\item Share.
\end{itemize}


% Please keep the Author Summary between 150 and 200 words
% Use first person. PLOS ONE authors please skip this step. 
% Author Summary not valid for PLOS ONE submissions.   
\section*{Author summary}
The way flipped classrooms are perceived and even practised by teachers is sometimes approximative. For example, while the Covid-19 pandemic has pushed many universities to adopt distance learning, flipped classrooms have often been mentioned as a solution in that context. This inducement maintains a confusion between flipped classrooms and distance learning that might be detrimental for both students and teachers. Moreover, embarking on a new pedagogical practice such as flipped classroom could be intimidating and time-consuming.

For these reasons, we have written these 10 simple rules to implement a flipped classroom. These 10 rules have been designed based on our own experiences but also rely heavily on the current scientific literature.


\linenumbers

% Use "Eq" instead of "Equation" for equation citations.
\section*{Introduction}

This article aims to share some tips for implementing a flipped classroom, with examples from biology and computational biology. 
The original idea of the flipped classroom involves reconsidering the type of work done by students autonomously (e.g., at home or in a library) and in class with the teacher, to provide a more active and effective learning experience for students \cite{bergmann_flip_2012,schell_flipping_2015}. 
Importantly, many practices are associated with the generic term 'flipped classroom'.
Here, we will not describe all the different forms, but we refer readers to comprehensive articles on the subject
\cite{bishop_flipped_2013,lebrun_vers_2016}.

Questions are often raised regarding the effectiveness of flipped classrooms. This question will not be addressed specifically herein, but we will make some general remarks on this topic. Performing an overall assessment of flipped classrooms is difficult, because this term refers to many widely differing practices. However, studies have shown positive effects with some protocols
\cite{casasola_can_2017, crouch_peer_2001, freeman_reply_2014}, including for students facing
difficulties \cite{lage_inverting_2000}. For the specific case of computational biology classes, 
Compeau has published detailed and positive feedback on this pedagogy \cite{compeau_establishing_2019}.
In addition, a series of meta-analyses by John Hattie \cite{hattie2014, hattie2018} suggest the substantial efficiency of flipped classrooms
\cite{chen_academic_2019,hew_flipped_2018,karagol_effect_2019,tan_effectiveness_2017}. 
A recent second-order meta-analysis has confirmed this conclusion and shown that flipped classrooms, compared with conventional classrooms,
significantly improve students’ cognitive and behavioral learning \cite{hew_does_2020}.
Finally, research on this topic is in an early stage, and several articles have called for more studies to assess
flipped classrooms 
\cite{abeysekera_motivation_2015,bishop_flipped_2013,hew_does_2020,lo_critical_2017}.


% Results and Discussion can be combined.
\section{10 simple rules}

The ten simple rules presented below are inspired by the French national conference on flipped classrooms, called CLIC, which has taken place 
in Paris since 2015. On the basis of the presentation of one of the authors (MB) at CLIC~2018, together with the audience’s remarks 
during the presentation and the recommendations from one of the references \cite{lo_critical_2017}, we have re-analyzed these rules from our practitioner's point of view and the perspective of the current literature.
These rules have also been presented and discussed during the 2020 CLIC conference (CLIC~2020).


\subsection{Rule 1. Invert time, and say it }

The basic principle of the flipped classroom involves considering the articulation of teaching and learning time \cite{bergmann_flip_2012,schell_flipping_2015}. Whereas schools and universities have long worked on a transmissive model, 
teaching no longer involves solely frontally led courses \cite{crouch_peer_2001}.

The spread of this practice has been accompanied by reflections on the nature of the tasks self-directed by learners. 
The goal is not giving more \cite{lo_critical_2017} but moving away from lower cognitive level classroom activities, 
and performing in-class tasks that require greater support from the teacher \cite{lebrun_vers_2016,anderson_taxonomy_2001,bloom_taxonomy_1956,sarawagi_flipped_2014} (Table 1). 
With this model, the transmission of knowledge occurs remotely, but knowledge assimilation can occur face-to-face, 
via active pedagogy sessions \cite{freeman_reply_2014}. 
The inverted class then enables maximization of the 'active' time in the classroom 
by moving the discovery of knowledge outside the classroom, without decreasing efficiency \cite{delozier_flipped_2017}.


% Place tables after the first paragraph in which they are cited.
\begin{table}[!ht]
\begin{adjustwidth}{-2.25in}{0in} % Comment out/remove adjustwidth environment if table fits in text column.
\centering
\caption{{\bf Activities performed autonomously and in the classroom with the teacher (left and middle columns) in relation to Bloom's taxonomy (right column)}}
\begin{tabular}{|l|l|l|}
\hline
{\bf Traditional classroom} & {\bf Flipped classroom} & {\bf Bloom's taxonomy} \\ \thickhline
In the classroom & Autonomous & Remember \\ \hline
In the classroom & Autonomous & Understand \\ \hline
In the classroom / Autonomous & Autonomous / in the classroom & Apply \\ \hline
Autonomous & In the classroom & Create, evaluate, analyze \\ \hline



\end{tabular}
%\begin{flushleft} Table notes Phasellus venenatis, tortor nec vestibulum mattis, massa tortor interdum felis, nec pellentesque metus tortor nec nisl. Ut ornare mauris tellus, vel dapibus arcu suscipit sed.
%\end{flushleft}
\label{table1}
\end{adjustwidth}
\end{table}


When preparing a flipped course, the teacher must design activities that can support 
the acquisition of elements of knowledge by autonomous students. For instance:

\begin{enumerate}

\item At the beginning of a course, prerequisites can be reactivated, or introductory concepts can be addressed.

\item Before a class, activities engage students with the material that they need to study. 
Activities can also prepare learners to confront a complex task.

\item Throughout the course, the teacher can also provide learners with a synthesis of the knowledge 
(through cheat sheets, quick reference cards or mnemonic devices), which can later be used to address complex tasks.

\item At the end of the course, students could work on a summary of the course.

\end{enumerate}

Flipped classrooms modify how teaching is traditionally performed. To support learners in this pedagogy, 
teachers must explain how the course is organized, such as through providing detailed instructions in
their syllabi \cite{grunert_obrien_course_2008}. One way to explicitly enforce what students 
are expected to do autonomously (e.g., at home or in a library) and in class is clearly defining pedagogical objectives, also termed 'learning outcomes'. 
While defining pedagogical objectives for flipped classrooms, teachers must keep in mind 
that activities performed autonomously by students should be related to the cognitive categories 'remember', 'understand' 
or `apply', as presented in Table 1. Eventually, teachers should also clearly state how the course assessments 
are conducted and specify the learning material.


\subsection{Rule 2. Identify (or create) resources for learning in autonomy}

That students have tasks to do autonomously means that they need some material to read, watch, or even interact with. 
Most teachers will first try to generate their own media. This process may be time consuming and sometimes discouraging, 
especially if the teacher aims to produce high-quality video content regularly. We advocate for the reuse of resources 
openly available online or in the physical world. These resources may be, for example, textbooks, books, web-documentaries, handouts, video clips, 
MOOC excerpts, or Wikipedia articles. The choice is made according to students' needs and the pedagogical objectives. 
Some selection criteria may include the resource’s trustworthiness, author's level of expertise, chapter or video length, 
presence of a detailed plan, or presence of companion exercises\ldots{} However, after careful assessment, the teacher will 
need to appropriate the selected material and, if needed, comment on or expand it. That is, teachers must integrate this resource into their pedagogical mindset.

For video resources specifically, Guo et al. \cite{guo_how_2014} have recommended choosing or creating short videos of 6 minutes or less, 
with a 'Khan academy' style, in which the teacher draws explanations on the screen. Other detailed recommendations are available 
on this subject \cite{mayer_cambridge_2005}. Many video resources are freely available online, from YouTube to open education
resources. Spend some time browsing them.

If you decide to create your own resources, try to implement the change gradually \cite{lo_critical_2017}. 
Instead of creating something new ‘from scratch’, you could use resources, protocols, and lesson plans 
shared by others, i.e., start with something that has worked elsewhere.

After the teaching material has been selected, there are many ways to share it with students:

\begin{itemize}

\item In a closed face-to-face system (for example, a handout distributed in class); no digital tool is required

\item In a closed remote online system, for example with a learning management system (LMS) such as Moodle, Blackboard, or Canvas, 
or shared cloud drives

\item In a remote open system, e.g., a class blog or a website

\end{itemize}

Regarding digital means to distribute teaching materials, some aspects should be emphasized:

\begin{itemize}

\item Ensuring access for all students is crucial \cite{lo_critical_2017}. 
Students must have the right equipment at home (internet connection, computer, or tablet) 
or access to a properly equipped library.

\item Learners must also be guided and supervised when using each new approach and tool \cite{lo_critical_2017}.

\item Although sharing material through an open system is not imperative, several positive effects can result. 
Resources are made available to other colleagues and teachers, and feedback from the community becomes possible.

\end{itemize}

\subsection{Rule 3. Be explicit in the acquisition of knowledge, and foster students' autonomy}

At the start of the course, be explicit about how the students will
work autonomously (see rule \#1). In addition to making a detailed syllabus, it is recommended to dedicate the first session to the setup of the flipped classroom. During this session, students can become acquainted with the teaching material and online tools with the help from the teacher or teaching assistants. Teachers also should be patient during the first weeks, so that the students become confident about the method and the tools.

Before class, teachers must ensure that the resources made available
have been consulted \cite{lo_critical_2017}, and the required tasks have been performed. 
Even if the activity performed autonomously by students is of a relatively low cognitive level, support is required to limit the socio-academic inequalities associated with work done outside the classroom \cite{rayou_faire_2010}. This outside-of-class support also provides indications that can be used by the teacher to adapt in-class sessions ~\cite{fidalgo2017apft}.
Students can complete several types of activities to support their acquisition of knowledge:

\begin{itemize}

\item Take quizzes (possibly self-corrected) on the proposed material. 
Use tools such as traditional LMS, Learning Apps, or Google Forms.

\item Write down open questions on paper or through digital means.

\item Participate in question and answer sessions through LMS or messaging applications (e.g., Slack, Mattermost, or Messenger\ldots).

\item Write a summary of the studied material. 
The produced summary can later be reviewed by the students’ peers.

\item Have students each list one thing that they found interesting and another that they want clarification on.

\item Collectively enrich a course trail or share the results of a search in a collaborative writing document. 
This activity is, however, subject to limited support and therefore must be followed by a class exchange session to make sure that the main elements are well understood.

\end{itemize}

The introduction of a flipped classroom is likely to be destabilizing or associated with misunderstandings about the objectives or about the methods used. Therefore, effective communication \cite{lo_critical_2017} and implementation of rule \#1 are crucial. 
Teachers require support from an efficient communication system (e-mail, social networks, or LMS). 
A forum could allow students to ask questions whose answers can then be provided by peers  or the teacher, before, during, or even after class. Defining an interaction protocol (who answers to whom, and when) can also be useful. 
Office hours, as implemented in the English-speaking academic world, could be highly helpful toward this goal.


\subsection{Rule 4. Explore active learning in class}

One key idea in flipping a class is making room for active learning during lectures \cite{delozier_flipped_2017}.
In active learning, students are involved and engage with the course material in a large range of activities, such as discussions, quizzes, or problem-solving\ldots{} 
Active learning, compared with traditional lecture-based teaching, has been found to improve student achievement \cite{freeman_reply_2014}, and some researchers have suggested that the performance of flipped classrooms 
could actually be explained by the use of active learning \cite{jensen_improvements_2015}.

Many types of active learning strategies can be explored, from summarizing book chapters to teaching part of the course \cite{fiorella_eight_2016}. Among the activities that can be recommended in class, a recent review \cite{delozier_flipped_2017} suggests the following:

\begin{itemize}

\item Quizzes with systems such as hands up, Plickers cards, clickers, 
or online voting systems (e.g., WooClap, Mentimeter, Kahoot, or Socrative)

\item Group discussions

\item Student presentations or teaching

\end{itemize}


More original methods can also be used, such as simulation games \cite{taly2019molecular} or working side by side 
with students on research questions \cite{mazzanti2017can}. 

Time outside the class should not be passive, 
however, because any activity can be adapted in a more active manner \cite{chi_icap_2014}.

In Table 1, we highlighted the interest in analyzing the tasks and activities that students must perform in the course of their learning. 
Most challenging tasks should be conducted in class with the guidance of the lecturer and through active learning. 
In the field of computational biology, learning a programming language and performing research projects are typical examples of learning objectives that are difficult to reach via traditional lecturing but strongly benefit from active learning. 
Interactive material such as Jupyter notebooks ~\cite{davies_using_2020,rule_ten_2019,davies2020} or the Rosalind auto-correcting exercise collection \cite{compeau_bioinformatics_2018} are valuable resources for learning how to program and discovering bioinformatics algorithms.
Engaging teams of students in open research challenges is also a unique opportunity for students to learn what a research project is about \cite{abdollahi2018meet}.

The number of pedagogical strategies proposed could be overwhelming for a teacher who would like to try flipped classrooms. 
However, teachers should keep in mind that errors are acceptable. 
Many iterations may be necessary to obtain an acceptable pedagogical scenario \cite{compeau_establishing_2019}. 
In that regard, one interesting and encouraging observation is the 'teacher effect', i.e. that enthusiastic teachers have a positive effect just through trying ~\cite{hattie_visible_2008}. 
We interpret this observation as a right for teachers to tinker.


\subsection{Rule 5. Develop skills of cooperation and sharing}

A potential advantage of flipped classrooms is that they can develop cooperation skills among students \cite{strayer_how_2012}. 
Cooperation has been observed to be an important parameter allowing the efficiency of flipped classrooms \cite{foldnes_flipped_2016}.

Typical examples of activities that can be performed in teams are editing shared documents, creating videos, or editing Wikipedia pages ~\cite{logan_ten_2010}. From this perspective, many tools now have a 'collaborative' feature to promote team production (e.g., Framapad, Prezi, Quizizz, Google Docs, and wikis). 
In the field of computational biology, editing and sharing computer code \cite{abdollahi2018meet} are possible with version control systems and online collaborative development platforms such as GitHub or Gitlab \cite{blischak_quick_2016}. Events such as coding hackathons \cite{garcia2020} can also be challenging opportunities.

To the extent that cooperation is a pedagogical objective, assessing the nature of the collaborative work is appropriate \cite{lebrun_vers_2016}. One way to do so is involving peers in evaluation \cite{lopez}.


\subsection{Rule 6. Evaluate}

Regular assessments are essential for the success of flipped classrooms. Assessment of the knowledge acquired during preparatory activities (i.e., autonomously) is useful not only to ensure that the activities have actually been achieved by the students but also to identify problems and address them in a targeted and immediate fashion. 
In class, voting systems such as Plickers, Quizizz, Wooclap, Kahoot, and Socrative can provide instant feedback on the knowledge acquired by students.

Beyond the assessment of students and their learning, evaluating the pedagogical protocols used is important. 
Among the indicators for evaluating a flipped classroom, not only the grades obtained by students but also broader criteria, such as absenteeism, conflict, autonomy, and cooperation, are interesting to consider.

Course evaluation surveys completed by students can also provide valuable feedback.

Because of the important need for further research in this field \cite{abeysekera_motivation_2015,bishop_flipped_2013,lo_critical_2017}, studies conducted by teachers could provide a basis for participatory research.

\subsection{Rule 7. Differentiate}

Lo and Hew have recommended individualized learning objectives ~\cite{lo_critical_2017}. Although flipped classrooms alone are not sufficient to perform differentiation, they can facilitate its implementation. All tasks and activities performed by a student provide a record of learning and progress. This tracking of learning enables problems to be identified and facilitates differentiation and remediation.

Flipped classrooms require teachers to be explicit (rule \#3) regarding the organization of the course (learning outcomes, assessments, teaching strategies, and learning materials\ldots). We believe that this aspect could benefit students with special needs. 
In addition, given that the transmissive portion of learning is studied individually and autonomously by students, this framework could in principle be adapted to student needs (e.g., by using different fonts or color schemes, or by enabling subtitles on videos).


\subsection{Rule 8. Take care of the logistics and your posture}

Engaging students to work on teaching material outside the classroom requires teachers to anticipate broad, inclusive, and seamless access to their material. Some technical solutions have already been provided above.

In class, some material will be needed to enable teamwork, such as flipcharts and sticky notes. 
To support learning situations in teams, the need to reorganize the space in the classroom quickly becomes apparent. 
This arrangement aims to facilitate exchanges between students but also illustrates changes in the posture of the teacher. 
Adoption of the flipped classroom approach is usually accompanied by the teacher’s reflection on the classroom layout, such that the teacher is no longer in front of or behind the students, but instead is close to or even in the middle of the students. 
The teacher then shifts 'from sage on the stage to guide on the side' \cite{king_sage_1993}. 
This organization often results in the establishment of work islands within which each team is invited to have an empty chair to allow the teacher to be 'beside the students'.


\subsection{Rule 9. Document your flipped classroom}

One advantage of a flipped classroom is bringing experiment and discovery back into the classroom. All trials and errors should be documented and written down, not only to track them but also to allow for retrospective analysis of the pedagogical approach used.

We advise teachers after each class to take a few minutes to write down what happened during the class, including both positive and negative points, and what could be improved upon in the next class. The students can be involved in that process. 
In our experience, adjustments proposed by students are often useful. Moreover, the process of consulting students helps nurture their intrinsic motivation \cite{oraif2018investigation,thai_impact_2017}.


\subsection{Rule 10. Share}

As stated in the introduction, there is not one but many flipped classrooms. Yours will probably not be like your colleague's flipped
classroom. After one or more years, your experience will be valuable. Share it!

You can publish your feedback as tweets, blog posts, or even research articles. To foster sharing and exchange, it is advisable to follow the FAIR guidelines, to make the media Findable, Accessible, Interoperable and Reusable ~\cite{garcia2020ten}. In particular, we encourage you to share your documents under open licenses such as Creative Commons Attribution (CC BY) or Creative Commons Attribution Share Alike (CC BY-SA).

Students’ productions can also be shared or published, then included in their portfolios, thus documenting their competencies. 
In the field of bioinformatics, a particularly appealing option is the sharing of code via development platforms such as GitHub or Gitlab \cite{blischak_quick_2016,abdollahi2018meet}.


\section*{Conclusion}

This article aims to gather some tips for implementing a flipped classroom. In addition to changing the teacher's posture ('from sage on the stage to guide on the side' \cite{king_sage_1993}), the inverted class approach changes the atmosphere and working conditions for both teachers and students.

Designing and setting up a flipped classroom is often accompanied by an increased teacher workload, which could be limited by adopting several strategies mentioned above: use of resources produced by others, automated corrections, or peer assignments \cite{guilbault_classe_2017}. Working in pedagogical teams or communities of practice is also a way of sharing the burden of work.

One aspect that we also highlighted is the need for more research on flipped classrooms \cite{hew_does_2020}. 
Therefore, enabling participatory research on the subject will be important to evaluate and document the multiple forms of flipped classrooms.


\section*{Acknowledgments}
{CLIC~2018 and CLIC~2020 participants}

\nolinenumbers

% Either type in your references using
% \begin{thebibliography}{}
% \bibitem{}
% Text
% \end{thebibliography}
%
% or
%
% Compile your BiBTeX database using our plos2015.bst
% style file and paste the contents of your .bbl file
% here. See http://journals.plos.org/plosone/s/latex for 
% step-by-step instructions.
% 

\bibliography{flipped}

\end{document}

